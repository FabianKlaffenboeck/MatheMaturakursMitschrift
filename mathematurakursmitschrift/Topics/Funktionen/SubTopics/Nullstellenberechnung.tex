\newpage
\subsection{Nullstellenberechnung}

Nullstellen sind Stellen an denen eine Funktion die $x$ Achse schneidet. Dort ist der Funktionswert $0$.
Diese könne einfach mit der ABC-Formel gelöst werden in dem dei Faktoren eisetzt.\\

Example:\\
\fboxrule=0.8pt \fcolorbox{black}{lightgray}{%
    \begin{tabular}[t]{@{}l@{}}
        $f(x)=0$   \\
        $-x^2+4=0$ \\
        $x^2=4$    \\
        $x=\pm 2$  \\
    \end{tabular}}\\

Example:\\
\fboxrule=0.8pt \fcolorbox{black}{lightgray}{%
    \begin{tabular}[t]{@{}l@{}}
        $3x^2+6x=0$   \\
        $3x(x-2)$     \\
        $x=\pm 2$     \\
        $x=0$         \\
        $x=2$         \\
        $N_1 = (0,0)$ \\
        $N_2 = (2,0)$ \\
    \end{tabular}}\\

Example:\\
\fboxrule=0.8pt \fcolorbox{black}{lightgray}{%
    \begin{tabular}[t]{@{}l@{}}
        $f(x)=-x^2+7x=0$                               \\
        $\textcolor{red}{x}\textcolor{blue}{(-x+7)}=0$ \\
        $x_1= \textcolor{red}{0}$                      \\
        $x_2=\textcolor{blue}{7}$                      \\
    \end{tabular}}\\

Example:\\
\fboxrule=0.8pt \fcolorbox{black}{lightgray}{%
    \begin{tabular}[t]{@{}l@{}}
        $f(x)=x^2-4x+3=0$                                 \\
        $x^2-4x\textcolor{red}{+4}=-3\textcolor{red}{+4}$ \\
        $(x\textcolor{red}{-2})^2 = 1$                    \\
        $x-2= \pm \sqrt{1}$                               \\
        $x = 2 \pm 1$                                     \\
        $x_1 = 3$                                         \\
        $x_2 = 1$                                         \\
    \end{tabular}}\\

Rechnungsweg im Taschenrechner $TI-82STATS$:\\
\begin{itemize}
    \item $2nd$ calc zerro
    \item Cursor 1 neben die nullstellen links plazieren 
    \item Cursor 2 neben die nullstellen rechts plazieren
\end{itemize}