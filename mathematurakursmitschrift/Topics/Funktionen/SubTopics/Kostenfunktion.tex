\newpage
\subsection{Kostenfunktion}

Die Grundform der Kostenfunktion ist $K(x)=k*x+F$.
Wobei...
\begin{itemize}
    \item ... $K(x)$ ist das Resultat der Gleichung ist also die Kosten.
    \item ... $k$ die Kosten pro Stück darstellt
    \item ... $x$ die Menge ist
    \item ... $F$ die Fixkosten darstellt
\end{itemize}

\hfill \break
Die Grundform der Erlösfunktion ist $E(x)=p*x$.
Wobei...
\begin{itemize}
    \item ... $E(x)$ ist das Resultat der Gleichung ist also der Erlös des Verkaufs
    \item ... $p$ der Verkaufspreis
    \item ... $x$ die Verkaufsmenge ist
\end{itemize}

\hfill \break
Die Grundform der Gewinnfunktion ist $G(x)=E(x)-K(x)$.
Wobei $G(x)$ der Gewinn ist.

\hfill \break
Example:\\
Die Fixkosten eines Betriebes betragen 140000€ pro Monat, die Produktionskosten 4€ pro Stück. Ein Stück wird zu 8€
verkauft.
Wie viel Stück müssen verkauft werden, damit ein Gewinn von mindestens 1000 000€ erzielt wird?:\\
\fboxrule=0.8pt \fcolorbox{black}{lightgray}{%
    \begin{tabular}[t]{@{}l@{}}
        $K(x)=4x+140000$             \\
        $E(x)=8x$                    \\
        $G(x)=E(x)-K(x)$             \\
        \hline
        $1000000=8x-(4x+140000)$     \\
        $1000000=8x-4x-140000$       \\
        $1000000=4x-140000$ /+140000 \\
        $1140000=4x$ / /4            \\
        $x=285000$                   \\
    \end{tabular}}