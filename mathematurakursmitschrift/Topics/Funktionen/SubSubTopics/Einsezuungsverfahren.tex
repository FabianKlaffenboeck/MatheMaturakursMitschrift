\subsubsection{Einsezuungsverfahren}

Beim Einsezuungsverfahren wird ein Term in den anderen eingesetzt.

\hfill \break
Example:\\
\fboxrule=0.8pt \fcolorbox{black}{lightgray}{%
    \begin{tabular}[t]{@{}l@{}}
        (1):$y=x+1$                                  \\
        (2):$+2x+y=4$ / $x+1$ wird in $y$ eingesetzt \\
        \\
        \\
        $+2x+x+1 = 4$                                \\
        $3x+1 = 4$ / -1                              \\
        $3x = 3$ / /3                                \\
        $x = 1$                                      \\
    \end{tabular}}

$$L=\{1,2\}$$