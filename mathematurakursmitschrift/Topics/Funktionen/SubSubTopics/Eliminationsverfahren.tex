\newpage
\subsubsection{Eliminationsverfahren}

Beim Eliminationsverfahren wird ein Teil des ersten Termes vom zweiten Term subtrahiert.

\hfill \break
Example:\\
\fboxrule=0.8pt \fcolorbox{black}{lightgray}{%
    \begin{tabular}[t]{@{}l@{}}
        (1):$y=x+1$ /$y$ wird elemeniert   \\
        (2):$y=-2x+4$ /$y$ wird elemeniert \\
        \\
        \\
        $0=x-(-2x)+1-4$                    \\
        $0=x+2x-3$                         \\
        $3=3x$ / /3                        \\
        $1=x$                              \\
    \end{tabular}}

$$L=\{1,2\}$$

\hfill \break
Um $x$ elemenieren zu können muss man zuerst bei beide Gleichungen die selbe Anzahl an $x$ erzeugen.

\hfill \break
Example:\\
\fboxrule=0.8pt \fcolorbox{black}{lightgray}{%
    \begin{tabular}[t]{@{}l@{}}
        (1):$4x + y = 16$          \\
        (2):$4x - 2y  = - 8 $ / /2 \\
        (2):$2x - y  = - 4$        \\
        \\
        \\
        $4x+y=16$                  \\
        $2x-y=-4$                  \\
        $6x=12$                    \\
        $x=2$                      \\
        \\
        \\
        In Ursprungsform Einsetzen \\
        $4*2+y=16$                 \\
        $8+y=16$ / -8              \\
        $y=8$                      \\
    \end{tabular}}

$$L=\{2,8\}$$