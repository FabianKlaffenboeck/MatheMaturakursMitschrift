\newpage
\section{Lineare regression}

Mit Hilfe einer Regression können aus Messwerten Zusammenhänge (Korrelationen) gefunden werden.
Dazu braucht man (mindestens) zwei Größen, zwischen welchen ein Zusammenhang vermutet wird, und geeignete Wertepaare.

\hfill \break
Beispiel:\\
Es kann von jedem Teilnehmer eines BRP-Kurses die Entfernung seines Wohnortes vom Kursort (Liste 1) und die Zeit (Liste 2), die er dafür benötigt, notiert werden.
Man kann vermuten, dass zwischen diesem Weg und dieser Zeit ein Zusammenhang besteht.

\hfill \break
Taschenrechner:\\
\begin{itemize}
    \item Stat 
    \item $\rightarrow$
    \item Edit
    \item Liste L1 und L2 eingeben 
    \item Stat 
    \item $\rightarrow$
    \item Calc
    \item Linreg: L1,L2
\end{itemize}

\newpage
\hfill \break
Wie gut passt ein lineares Modell?\\
Für eine lineare Regression gibt es eine Größe, die angibt, wie gut das Lineare Modell passt.
Das ist der Korrelationskoeffizient (nach Pearson) r.
Sein Wert liegt immer zwischen -1 und 1. Er gibt an, ob ein linearer Zusammenhang vermutet werden kann, besteht oder eher ausgeschlossen werden kann.
Ob die Größen einander wirklich beeinflussen, kann aber nie mit Sicherheit bestimmt werden.

\hfill \break
\begin{itemize}
    \item $|r|\approx 1$ Ist der Betrag des Korrelationskoeffizienten nahe bei 1, so passt das lineare Modell sehr gut.
    \item $|r|\approx 0$ Ist der Korrelationskoeffizient sehr klein, nahe bei 0, so liegt vermutlich kein linearer Zusammenhang vor.
    \item $r > 0$ Ist der Korrelationskoeffizient positiv, so ist die Trendgerade steigend. Man spricht von einer positiven Korrelation.
    Wenn der Wert einer Größe zunimmt, so steigt auch der Wert der anderen Größe. (z.B.: Wer eine längere Strecke zurücklegen muss, braucht auch mehr Zeit.)
    \item Ist der Korrelationskoeffizient negativ, so ist die Trendgerade fallend. Man spricht von einer negativen Korrelation.
    Wenn der Wert einer Größe zunimmt, so sinkt der Wert der anderen Größe. (z.B.: Wenn der Preis eines Produktes steigt, so wird die verkaufte Menge sinken.)
    \item $r = 1$ oder $r=0$ Erreicht der Korrelationskoeffizient die Werte 1 oder -1 exakt, so handelt es sich um eine vollkommene Korrelation.
    Das heißt, dass man nicht eine möglichst gute Gerade gefunden hat, sondern dass alle Punkte exakt auf dieser Geraden liegen.
\end{itemize}

\hfill \break
Am Taschenrechner erhält man den Korrelationskoeffizienten automatisch bei der Berechnung der linearen Regression, wenn er aktiviert wurde.
Um dies zu tun braucht man aus dem Catalog (2nd: 0) den Befehl DiagnosticOn.
Wird dieser ausgewählt und mit Enter bestätigt, wird in weiterer Folge immer der Korrelationskoeffizient automatisch mitberechnet.
(Auch wenn der Taschenrechner zwischendurch ausgeschaltet wird, bleibt die Aktivierung erhalten.)


\newpage
\hfill \break
Gibt es nur lineare Trendlinien?\\
Als Modelle für Regressionen eignen sich verschieden Funktionstypen.
Zur Anwendung kommen insbesondere Polynomfunktionen bis zum Grad 4 und Logarithmus- und Exponentialfunktionen.
Den Korrelationskoeffizienten nach Pearson gibt es allerdings nur bei einer linearen Regression.

\hfill \break
Hier sieht man verschiedene Trendlinien für die gleichen Wertepaare.
Zuerst wurde eine Polynomfunktion 3. Grades als Modell gewählt.
Anschließend eine Exponentialfunktion und zum Schluss eine Logarithmusfunktion.