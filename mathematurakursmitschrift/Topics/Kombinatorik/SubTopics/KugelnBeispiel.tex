\newpage
\subsection{Kugeln Beispiel}

Aus einer Urne mit 4 weißen, 3 schwarzen 1 roten Kugeln wird dreimal ohne zurücklegen gezogen. Man berechne die
Wahrscheinlichkeit des folgenden Ergebnisses:

\begin{enumerate}
    \item Keine der gezogenen Kugeln ist rot
    \item Es kommen genau zwei weiße Kugeln vor
    \item Alle Kugeln haben dieselbe Farbe
    \item Jede Farbe kommt vor
\end{enumerate}


\hfill \break
\begin{enumerate}
    \item $P(\lnot r,\lnot r,\lnot r) = \frac{7}{8}*\frac{6}{7}*\frac{5}{6} = \frac{5}{8}$
    \item
          $P(w,w,\lnot w) = \frac{4}{8}*\frac{3}{7}*\frac{4}{6}$\\
          $P(w,\lnot w,w) = \frac{4}{8}*\frac{4}{6}*\frac{3}{7}$\\
          $P(\lnot w,w,w) = \frac{4}{6}*\frac{4}{8}*\frac{3}{7}$\\
          kann auf $3*\frac{4}{8}*\frac{3}{7}*\frac{4}{6}$ zusammnegefasst werden.
    \item $P(w,w,w)+P(s,s,s)+P(r,r,r) = \frac{4}{8}*\frac{3}{7}*\frac{2}{6}+\frac{3}{8}*\frac{2}{7}*\frac{1}{6}+\frac{1}{8}*\frac{0}{7}*\frac{0}{6}$
    \item $P(s,w,r)+P(s,r,w)+P(s,r,w)+P(w,s,r)+P(w,r,s)+P(r,s,w)+P(r,w,s) = 6*\frac{3}{8}*\frac{4}{7}*\frac{1}{6}$
\end{enumerate}