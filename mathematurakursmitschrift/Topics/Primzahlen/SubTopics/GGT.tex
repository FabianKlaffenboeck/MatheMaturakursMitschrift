\newpage
\subsection{Größter gemeinsamer Teiler (ggt)}

Das GGT wird mittles Primfaktorzerlegung ermittelt.\\
Die Zahlen die in beiden Zerlegungen vorhanden sind werden Zusammen-multipliziert.\\

\hfill \break
Example: Ermittling des ggt von 36 und 60\\
\fboxrule=0.8pt \fcolorbox{lightgray}{lightgray}{%
    \begin{tabular}{ c|c||c|c}
        36 & \textcolor{red}{2}   & 60 & \textcolor{red}{2}   \\
        18 & \textcolor{green}{2} & 30 & \textcolor{green}{2} \\
        9  & \textcolor{blue}{3}  & 15 & \textcolor{blue}{3}  \\
        3  & 3                    & 5  & 5                    \\
        1  &                      & 1  &                      \\
    \end{tabular}}
$ggt(36,60) = \textcolor{red}{2}*\textcolor{green}{2}*\textcolor{blue}{3} = 12$