\newpage
\subsection{Erwartungswert}


Am Beispiel des Würfeln eines Würfels:\\

\begin{itemize}
    \item $n=1800$
    \item $P(\sigma) = \frac{1}{6}$
    \item $P(\lnot 6) = \frac{5}{6}$
\end{itemize}

Der Erwartungswert kann mit der Formel $E(x)=1800 * \frac{1}{6} = 300$\\
$\sigma = \sqrt{1800*\frac{1}{6}*\frac{5}{6}} = 15.8 \simeq 16$

\hfill \break
Struebereich: $300 \pm 16 = \left[285,316\right]$

\hfill \break
Der Bereich um $\mu$ (MU) ist der Streubereich, es kann erwarted werden das im Bereich $\left[285,316\right]$ ein 6er gefunden wird.

\hfill \break
\begin{itemize}
    \item $\sigma_1 = 285$
    \item $\sigma_2 = 316$
\end{itemize}

\hfill \break
\begin{tikzpicture}
    \begin{axis}[every axis plot post/.append style={
                    mark=none,domain=250:350,samples=50,smooth},
            axis x line*=bottom,
            axis y line*=left,
            enlargelimits=upper]
        \addplot {gauss(300,16)};
    \end{axis}
\end{tikzpicture}