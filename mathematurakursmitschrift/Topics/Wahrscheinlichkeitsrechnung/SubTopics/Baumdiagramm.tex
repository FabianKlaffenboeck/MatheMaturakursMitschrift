\newpage
\subsection{Baumdiagramm}


\hfill \break
\subsubsection{Baumdiagram von zweimaligem werfen einer Münze}

2x werfen einer Münze

\hfill \break
\begin{forest}
    for tree={l=50, delay={edge label/.wrap value={node[midway, font=\sffamily\scriptsize, auto]{#1}}}}
    [Erster Münzwurf
    [Kopf, edge label=$\frac{1}{2}$ [Kopf, edge label=$\frac{1}{2}$] [Zah, edge label=$\frac{1}{2}$]]
    [Zahl, edge label=$\frac{1}{2}$ [Kopf, edge label=$\frac{1}{2}$] [Zah, edge label=$\frac{1}{2}$]]
    ]
\end{forest}


\hfill \break
\begin{enumerate}
    \item $P(K,Z) = \frac{1}{2}* \frac{1}{2} =  \frac{1}{4} = 25\%$
    \item $P((K,Z),(Z,K)) = \frac{1}{2}* \frac{1}{2} +\frac{1}{2}* \frac{1}{2} =  \frac{1}{4}+\frac{1}{4} = \frac{1}{2} = 50\%$
\end{enumerate}

\hfill \break
\subsubsection{Ziehen ohne zurücklegen}

2x ziehen ohne zurücklegen

\hfill \break
\begin{forest}
    for tree={l=50, delay={edge label/.wrap value={node[midway,  auto]{#1}}}}
    [3 Weiß und 2 Schwarz
        [Weiß, edge label=$\frac{3}{5}$ [Weiß, edge label=$\frac{2}{4}$] [Schwarz, edge label=$\frac{2}{4}$]]
        [Schwarz, edge label=$\frac{2}{5}$ [Weiß, edge label=$\frac{3}{4}$] [Schwarz, edge label=$\frac{1}{4}$]]
    ]
\end{forest}

\hfill \break
\begin{itemize}
    \item $P(W,S) = \frac{3}{5}*\frac{2}{4} = \frac{1}{5} = 30\%$
    \item $P(W,W) = \frac{3}{5}*\frac{2}{4} = \frac{1}{5} = 30\%$
    \item $P(S,S) = \frac{2}{5}*\frac{1}{4} = \frac{1}{10} = 10\%$
\end{itemize}

\hfill \break
Praktisches Beispiel:\\
Ein Kaufhaus bezieht Hemden von drei verschiedenen Firmen. Von 3000 Stück stammen 900 von der Firma A, 1500 von
der Firma Bund der Rest von der Firma C. Statistische Untersuchungen haben ergeben, dass Waren der Firma A mit
einer Wahrscheinlichkeit von $10\%$, der Firma B mit $20\%$ und der Firma C mit $15\%$ fehlerhaft sind.
\begin{enumerate}
    \item Wie viel Stück der gelieferten Hemden sind fehlerhaft
    \item Das Hemd, das gekauft wurde, ist in Ordnung. Mit welcher Wahrscheinlichkeit wurde es von der Firma B erzeugt?
    \item Wie viel Stück müssen bei einer Qualitätskontrolle geprüft werden, damit mit einer Wahrscheinlichkeit von wenigstens $90\%$ mindestens ein fehlerhaftes Stück dabei ist?
\end{enumerate}


\hfill \break
Baumdiagram:\\
\begin{forest}
    for tree={l=50, delay={edge label/.wrap value={node[midway, font=\sffamily\scriptsize, auto]{#1}}}}
    [Hemden
        [A, edge label=$\frac{3}{10}$ [$f$, edge label=$0.1$][$\lnot f$, edge label=$0.9$]]
        [B, edge label=$\frac{1}{2}$ [$f$, edge label=$0.2$][$\lnot f$, edge label=$0.8$]]
        [C, edge label=$\frac{1}{5}$ [$f$, edge label=$0.15$][$\lnot f$, edge label=$0.85$]]
    ]
\end{forest}


\hfill \break
Example Lösung:\\
\begin{enumerate}
    \item $P(f) = 0.3*0.1+0.5+0.2+0.2+0.2*0.15=0.16 \longrightarrow 3000/0.16 = 480$ Stück sind fehlerhaft.
    \item $P(\textrm{OK und Firma B}) = 0.5*0.8=0.4 = 40\%$
    \item $P(\textrm{bin 1 fehlerhafzes}) = 1-P(\textrm{kein fehlerhaftes}\geq 0.9)$\\ Rechenweg: $1-0.84^n \geq 0.9 \rightarrow 0.1 \geq 0.85^n \rightarrow \frac{ln(0.1)}{ln(0.85)} \leq n \rightarrow 13.2 \leq n$
\end{enumerate}
