\newpage
\section{Wahrscheinlichkeitsrechnung}

Wichtige Zeichen, Sybole udn Regeln:\\
\begin{itemize}
    \item $\lnot$ bedeuted soviel wie "Verneinung" oder "nicht"
    \item "oder" bedeuted soviel wie + in der Rechnung später
    \item "und" bedeuted soviel wie * in der Rechnung später
\end{itemize}


\hfill \break
Grundidee:\\
Zufallsexperiment ist das Experiment das Dem Zufall ausgesetzt ist.\\ 
Den möglichen Ausgang nennt man Ereigniss.\\
Die Menge aller Ereignisse nennt man ereignisraum Zeichen: $\Omega$.\\
$\Omega = {1,2,3,4,5,6}$ zb wäre der ereignisraum beim Würfeln eines Würfels (einmalig).

\hfill \break
Zufallsexperimente Beispiele:
\begin{itemize}
    \item Lotto
    \item Joker
    \item Münzwurf
    \item Würfeln
\end{itemize}

\hfill \break
Das Laplace Experiment:\\
Bei diesem Experiment haben alle Ereignisse die selbe Wahrscheinlichkeit:\\
zb.: das Werfen eines Würfels $P(6er) = \frac{1}{6}$ das $P$ steht dabei für den griechischen ausdruck Probabilitas.\\
\begin{itemize}
    \item $P(6er) = \frac{1}{6}$
    \item $P(4er) = \frac{1}{6}$
    \item $P(1er) = \frac{1}{6}$
\end{itemize}

\hfill \break
Formel: $P(E) = \frac{\textrm{Anzahl der günstigen Fälle}}{\textrm{Anzahl der möglichen Fälle}}$

\hfill \break
Die Wahrscheinlichkeit eiens Ereignisses ist immer zwischen 0 und 1 beziungsweise $0\%$ und $100\%$\\
\begin{itemize}
    \item $P(E)=0$ unmögliches Ereigniss
    \item $P(E)=1$ sicherres Ereigniss
\end{itemize}

\hfill \break
\newpage
\subsection{Baumdiagramm}


\hfill \break
\subsubsection{Baumdiagram von zweimaligem werfen einer Münze}

2x werfen einer Münze

\hfill \break
\begin{forest}
    for tree={l=50, delay={edge label/.wrap value={node[midway, font=\sffamily\scriptsize, auto]{#1}}}}
    [Erster Münzwurf
    [Kopf, edge label=$\frac{1}{2}$ [Kopf, edge label=$\frac{1}{2}$] [Zah, edge label=$\frac{1}{2}$]]
    [Zahl, edge label=$\frac{1}{2}$ [Kopf, edge label=$\frac{1}{2}$] [Zah, edge label=$\frac{1}{2}$]]
    ]
\end{forest}


\hfill \break
\begin{enumerate}
    \item $P(K,Z) = \frac{1}{2}* \frac{1}{2} =  \frac{1}{4} = 25\%$
    \item $P((K,Z),(Z,K)) = \frac{1}{2}* \frac{1}{2} +\frac{1}{2}* \frac{1}{2} =  \frac{1}{4}+\frac{1}{4} = \frac{1}{2} = 50\%$
\end{enumerate}

\hfill \break
\subsubsection{Ziehen ohne zurücklegen}

2x ziehen ohne zurücklegen

\hfill \break
\begin{forest}
    for tree={l=50, delay={edge label/.wrap value={node[midway,  auto]{#1}}}}
    [3 Weiß und 2 Schwarz
        [Weiß, edge label=$\frac{3}{5}$ [Weiß, edge label=$\frac{2}{4}$] [Schwarz, edge label=$\frac{2}{4}$]]
        [Schwarz, edge label=$\frac{2}{5}$ [Weiß, edge label=$\frac{3}{4}$] [Schwarz, edge label=$\frac{1}{4}$]]
    ]
\end{forest}

\hfill \break
\begin{itemize}
    \item $P(W,S) = \frac{3}{5}*\frac{2}{4} = \frac{1}{5} = 30\%$
    \item $P(W,W) = \frac{3}{5}*\frac{2}{4} = \frac{1}{5} = 30\%$
    \item $P(S,S) = \frac{2}{5}*\frac{1}{4} = \frac{1}{10} = 10\%$
\end{itemize}

\hfill \break
Praktisches Beispiel:\\
Ein Kaufhaus bezieht Hemden von drei verschiedenen Firmen. Von 3000 Stück stammen 900 von der Firma A, 1500 von
der Firma Bund der Rest von der Firma C. Statistische Untersuchungen haben ergeben, dass Waren der Firma A mit
einer Wahrscheinlichkeit von $10\%$, der Firma B mit $20\%$ und der Firma C mit $15\%$ fehlerhaft sind.
\begin{enumerate}
    \item Wie viel Stück der gelieferten Hemden sind fehlerhaft
    \item Das Hemd, das gekauft wurde, ist in Ordnung. Mit welcher Wahrscheinlichkeit wurde es von der Firma B erzeugt?
    \item Wie viel Stück müssen bei einer Qualitätskontrolle geprüft werden, damit mit einer Wahrscheinlichkeit von wenigstens $90\%$ mindestens ein fehlerhaftes Stück dabei ist?
\end{enumerate}


\hfill \break
Baumdiagram:\\
\begin{forest}
    for tree={l=50, delay={edge label/.wrap value={node[midway, font=\sffamily\scriptsize, auto]{#1}}}}
    [Hemden
        [A, edge label=$\frac{3}{10}$ [$f$, edge label=$0.1$][$\lnot f$, edge label=$0.9$]]
        [B, edge label=$\frac{1}{2}$ [$f$, edge label=$0.2$][$\lnot f$, edge label=$0.8$]]
        [C, edge label=$\frac{1}{5}$ [$f$, edge label=$0.15$][$\lnot f$, edge label=$0.85$]]
    ]
\end{forest}


\hfill \break
Example Lösung:\\
\begin{enumerate}
    \item $P(f) = 0.3*0.1+0.5+0.2+0.2+0.2*0.15=0.16 \longrightarrow 3000/0.16 = 480$ Stück sind fehlerhaft.
    \item $P(\textrm{OK und Firma B}) = 0.5*0.8=0.4 = 40\%$
    \item $P(\textrm{bin 1 fehlerhafzes}) = 1-P(\textrm{kein fehlerhaftes}\geq 0.9)$\\ Rechenweg: $1-0.84^n \geq 0.9 \rightarrow 0.1 \geq 0.85^n \rightarrow \frac{ln(0.1)}{ln(0.85)} \leq n \rightarrow 13.2 \leq n$
\end{enumerate}

\hfill \break
\newpage
\subsection{Experimente}

\hfill \break
\subsubsection{Würfeln mit einem Würfel}

\hfill \break
Würfeln mit einem Würfel (einmalig):\\
$\Omega = {1,2,3,4,5,6}$\\
mögliche Ereignisse:\\
\begin{itemize}
    \item ein 6er Würfeln: $A={6}$: $P(\textrm{in 6er Würfeln}) \Rightarrow \frac{1}{6} \Rightarrow 16.67\%$
    \item gerade Augenzahl: $A={2,4,6}$: $P(\textrm{gerade Augenzahl}) \Rightarrow \frac{3}{6} \Rightarrow 50\%$
    \item Augenzahl kleiner als 3: $A={1,2}$: $P(\textrm{Augenzahl < 3}) \Rightarrow \frac{2}{3} \Rightarrow 33.\overline{3}\%$
\end{itemize}

\hfill \break
werfen einer Münze (zweimal):\\
$\Omega = {KK,KZ,ZK,ZZ}$\\
mögliche Ereignisse:\\
\begin{itemize}
    \item 2x Kopf: $A={KK}$
    \item 1x Kopf und 1x Zahm: dabei gillt das Reihenfolgengesetz volgende Zustände sind möglich: \begin{itemize}
              \item Reihung ist egal: $A={KZ,ZK}$
              \item Reihung ist nicht egal: $A={KZ}$
          \end{itemize}
\end{itemize}


\newpage
\subsubsection{Wie oft mindestens würfeln}

\hfill \break
Wie oft mus man mindestens würfeln, um mit mindestens $95\%$ Wahrscheinlichkeit mindestens einen 6er zu würfeln?\\
$P(6er) = \frac{1}{6}$\\
$P(x*6er) = \frac{1}{6}^x$\\
$\lnot P(6er) = \frac{5}{6}$\\

\hfill \break
$x..$ = Anzahl der 6er

\hfill \break
\fboxrule=0.8pt \fcolorbox{black}{lightgray}{%
    \begin{tabular}[t]{@{}l@{}}

        $1-(\frac{5}{6})^n \geq 0.95$                                                          \\
        $0.25 \geq (\frac{5}{6})^n$ /  *ln   durch das teilen dreht sich das $\geq$ Zeichen um \\
        $ln(0.05) \geq ln(\frac{5}{6})^n$                                                      \\
        $ln(0.05) \geq n*\frac{5}{6}$ / $\div\frac{5}{6} $                                     \\
        $\frac{ln(0.05)}{ln(\frac{5}{6})} \leq n $                                             \\
        $16.43 \leq n$                                                                         \\
    \end{tabular}}
\hfill \break
\subsection{Gegenwahrscheinlichkeit}

\hfill \break
1 6er würfeln: $P(A) = \frac{1}{6}$\\
$\lnot$ 6er würfeln: $P(A) = 1-\frac{1}{6} = \frac{5}{6}$

\hfill \break
Bei mindestens 1-mal ist die Gegenwahrscheinlichkeit einfach zu berechnen, da das Gegenereignis einmal ist $E \geq 1$\\
Gegenwahrscheinlichkeit: $E < 1 \textrm{ bei } E = 0$
\hfill \break
\newpage
\subsection{Erwartungswert}


Am Beispiel des Würfeln eines Würfels:\\

\begin{itemize}
    \item $n=1800$
    \item $P(\sigma) = \frac{1}{6}$
    \item $P(\lnot 6) = \frac{5}{6}$
\end{itemize}

Der Erwartungswert kann mit der Formel $E(x)=1800 * \frac{1}{6} = 300$\\
$\sigma = \sqrt{1800*\frac{1}{6}*\frac{5}{6}} = 15.8 \simeq 16$

\hfill \break
Struebereich: $300 \pm 16 = \left[285,316\right]$

\hfill \break
Der Bereich um $\mu$ (MU) ist der Streubereich, es kann erwarted werden das im Bereich $\left[285,316\right]$ ein 6er gefunden wird.

\hfill \break
\begin{itemize}
    \item $\sigma_1 = 285$
    \item $\sigma_2 = 316$
\end{itemize}

\hfill \break
\begin{tikzpicture}
    \begin{axis}[every axis plot post/.append style={
                    mark=none,domain=250:350,samples=50,smooth},
            axis x line*=bottom,
            axis y line*=left,
            enlargelimits=upper]
        \addplot {gauss(300,16)};
    \end{axis}
\end{tikzpicture}
