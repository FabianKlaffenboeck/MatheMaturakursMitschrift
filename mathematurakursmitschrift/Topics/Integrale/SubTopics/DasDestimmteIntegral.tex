\newpage
\subsection{Das bestimmte Integral}

Die Ableitung der Stammfunktion ergibt die Funktion selbst. $\rightarrow F'(x)=f(x)$

\hfill\break
\begin{itemize}
    \item Bestimme die Stammfunktion der Funktion $f(x)=2x$
    \item Welche Funktion ergibt abgeleitet $f(x)=2x \rightarrow F(x)=x^2$
    \item Weil $\rightarrow F'(x) = 2x = f(x)$
\end{itemize}

Eine Ableitungsregel besagt, dass eine Konstante beim Ableiten wegfällt.
Aus diesem Grund ist die oben angegebene Lösung nur eine von unendlich vielen, denn auch z.B. $F(x)=x^2+3$ und $F(x)=x^2-9$ sind Stammfunktionen von $f(x)=2x$.
Da sich die einzelnen Stammfunktionen nur durch eine Konstante $C$ unterscheiden, schreiben wir $F(x)=x^2+C$