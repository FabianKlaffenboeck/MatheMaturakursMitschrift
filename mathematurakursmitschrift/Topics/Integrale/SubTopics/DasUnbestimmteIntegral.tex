\newpage
\subsection{Das unbestimmte Integral}

Die Fläche wird immer von Nullstelle zu Nullstelle berechnet weil wen man versucht die rechn ung auf einmal
mit den taschenrechner zu lösen wird alles unter dem Nullpunkt von allem größer null subtrahieren.


$$A=\int\limits_1^7 3dx=3x+c |_1^7 = \textcolor{red}{3*7+x}-\textcolor{blue}{(3*1+c)} = 21-3 = 18$$
\begin{itemize}
    \item \textcolor{red}{obere Gänze}
    \item \textcolor{blue}{unterre Gänze}
\end{itemize}


\hfill \break
Example:\\
\fboxrule=0.8pt \fcolorbox{lightgray}{lightgray}{%
    \begin{tabular}{ |c|}
        \hline
        GrundFormel: $f(x) = -x^2+4$                                                                                                                                                                \\
        $F(x) = -\frac{x^3}{3} +4x$                                                                                                                                                                 \\
        Integral: $\int\limits_{-2}^2 f(x)dx$                                                                                                                                                       \\
        $\int\limits_{-2}^2 f(x)dx = 2*\int\limits_0^2 f(x)dx = 2*(-\frac{x^2}{3}+4 |_0^2) = 2*(\frac{2}{3}+4*2(-\frac{0^3}{3}+4*0)=2*(-\frac{8}{3}+8) = 2* \frac{16}{3} = \frac{32}{3})$ \\
        \hline
    \end{tabular}}\\

\hfill \break
\begin{itemize}
    \item Gegben: $f(x)=3$
    \item Gesucht: Flächeninhalt ($f(x)$) von $x_1=1$ bis $x_2=7$
\end{itemize}


\hfill \break
Mit dem Taschenrechner: Rechnerrisch:
\begin{enumerate}
    \item Math $\uparrow$
    \item a: fnInt
    \item fnint(3,x,\textcolor{red}{1},\textcolor{blue}{7}) $\textcolor{red}{1=x_1}$ $\textcolor{blue}{7=x_2}$
\end{enumerate}

\hfill \break
Mit dem Taschenrechner: Grafisch:
\begin{enumerate}
    \item $y_1$ = 3
    \item $2_{nd}$ calc 7. f(x)dx
\end{enumerate}

