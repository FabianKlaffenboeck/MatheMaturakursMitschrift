\newpage
\subsection{Wie kann man mit Vektoren rechnen?}

\hfill \break
Wenn man zwei Vektoren $\vec{a} ? \binom{a_1}{a_2}$ und $\vec{b} ? \binom{b_1}{b_2}$ addiert, so hängt man sie einfach hintereinander.
$$\vec{a}+\vec{b} = \binom{a_1}{a_2} + \binom{b_1}{b_2} = \binom{a_1+b_1}{a_2+b_2}$$

\hfill \break
Wenn es sich um Entfernungen zweier Punkte handelt, so geht man (auf der Schatzkarte) beide Wege bzw. als Summe den direkten Weg.

\hfill \break
Handelt es sich um physikalische Größen, so bildet man den resultierenden Vektor.
Man erhält ihn durch ein Parallelogramm. Man erhält dasselbe Ergebnis wie in ersterem Fall.
Wenn beide Kräfte $\vec{F_1}$ und $\vec{F_2}$ wirken, so kann man sie beide durch die resultierende Kraft $\vec{F}$ ersetzen.
Sie hat die gleiche Wirkung, wie die beiden anderen Kräfte zusammen. (Kräfteparallelogramm)

\hfill \break
Subtrahiert man einen Vektor, so wird eigentlich sein Gegenvektor addiert.
$$\vec{a} - \vec{b} = \binom{2}{4}-\binom{5}{2} = \binom{2}{4} + \binom{-5}{-2}-\binom{-3}{2}$$

\hfill \break
Beim Addieren von zwei Vektoren entsteht immer ein neuer Vektor. Dieser kann länger oder auch kürzer als die einzelnen ursprünglichen Vektoren sein.
\begin{itemize}
    \item Wenn man zwei Wege hintereinander geht, hätte man auch eine Abkürzung statt dessen gehen können und der Weg wäre kürzer gewesen.
    \item Wenn Kräfte nicht in die gleiche Richtung wirken, sondern gegeneinander, so wird die resultierende Kraft kleiner sein.
\end{itemize}

\hfill \break
Für allgemeine Größen ist die Addition von Vektoren ideal, weil mehrere Werte gleichzeitig addiert werden können.