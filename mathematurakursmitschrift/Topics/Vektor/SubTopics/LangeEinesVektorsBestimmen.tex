\newpage
\subsection{Wie kann die Lage von Vektoren bestimmt werden?}

\hfill \break
Vektoren können auch multipliziert werden.
Bildet man das Skalarprodukt zweier Vektoren $\vec{a} = \binom{a_1}{a_2}$ und $\vec{b} = \binom{b_1}{b_2}$, so ist das Ergebnis ein Skalar (eine Zahl).
$$\vec{a} * \vec{b} = \binom{a_1}{a_2} * \binom{b_1}{b_2} = a_1 * b_1 + a_2 * b_2$$

\hfill \break
Beim Skalarprodukt wird einer der Vektoren auf den anderen "normal" projiziert.
Hier sieht man, wie der Vektor $\vec{d}$ auf den Vektor $\vec{a}$ projiziert wird.
Es ergibt sich der Vektor $\vec{b_a}$.
Das Skalarprodukt berechnet dann das Produkt der Längen von $\vec{a}$ und von $\vec{b_a}$.\\
Verwendet wird das Skalarprodukt, wenn man die Lage von Vektoren zueinander untersuchen möchte.

\hfill \break
Das Skalarprodukt von Vektoren, die normal aufeinander stehen, ist 0.
(Die Projektion liefert einen Vektor mit Länge 0.)
$$\vec{c} * \vec{d} = \binom{4}{2} * \binom{1}{-2} = 4 * 1 + 2 * (-2) = 0$$

\hfill \break
Skalarprodukte können positiv oder negativ sein, das liegt an dem Winkel, den die Vektoren miteinander einschließen.
Bei einem negativen Ergebnis ist der Winkel stumpf, bei einem positiven Ergebnis ist er spitz. Ist das Ergebnis O ist der Winkel 90°.

\hfill \break
\begin{itemize}
    \item $\vec{e} = \binom{3}{-2}$
    \item $\vec{f} = \binom{-1}{3}$
    \item $\vec{g} = \binom{4}{6}$
\end{itemize}

\hfill \break
$$\vec{e} * \vec{f} = 3 * (-1) + (-2) * 3 = -9$$
$$\vec{e}*\vec{f} = 3 * 4 + (-2) * 6 = 0$$
$$\vec{f} * \vec{g} = (-1) * 4 + 3 * 6 = 14$$