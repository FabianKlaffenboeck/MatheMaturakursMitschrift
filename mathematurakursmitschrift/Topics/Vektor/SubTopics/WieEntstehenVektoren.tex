\newpage
\subsection{Wie entstehen Vektoren? Wozu dienen Vektoren?}

\hfill \break
Vektoren können Wege beschreiben, wie man von einem Punkt zu einem anderen gelangt, wie bei einer Schatzkarte (3 Schritte nach Ost und 4 Schritte nach Nord).
Dazu verwendet man die Koordinaten der Punkte und subtrahiert die des Anfangspunktes vom Endpunkt (Spitze - Schaft).

\hfill \break
Verwendet man die Punkte in Vektorschreibweise, so nennt man sie Ortsvektoren.
Sie entsprechen der Verbindung vom Ursprung zum Punkt.

\hfill \break
Vektoren können auch physikalische Größen beschreiben, die eine Richtung aufweisen. Dabei kann es sich z.B. um
Geschwindigkeiten handeln oder Kräfte.

\hfill \break
Der Vektor $\vec{v}$ beschreibt eine Geschwindigkeit in Richtung Nordost.
Die Länge des Vektors entspricht dabei der Betrag der Geschwindigkeit.
Je länger der Vektor, desto größer die Geschwindigkeit.

\hfill \break
Die Vektoren $\vec{F_1}$ und $\vec{F_2}$ können Kräfte beschreiben, die zwei Freunde beim Seilziehen aufwenden.
(Sie ziehen entgegen-gesetzt.) Wieder gilt, je größer die Kraft, desto länger der Vektor.
Daher wird der mit der Kraft $\vec{F_1}$gewinnen.

\hfill \break
Vektoren können auch allgemeine Größen darstellen:\\
\begin{itemize}
    \item X ... Anzahl der gekauften Kuchen
    \item Y ... Anzahl der gekauften Torten
\end{itemize}
Der Vektor $\vec{s}$ besagt daher, dass 3 Kuchen und 5 Torten gekauft wurden.
Bei Vektor $\vec{t}$ sind es 4 Kuchen und eine Torte.