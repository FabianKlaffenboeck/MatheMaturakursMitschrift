\newpage
\subsection{Wie kann die Länge eines Vektors berechnet werden?}

\hfill \break
Die Länge (den Betrag) eines Vektors $\vec{a} = \binom{a_1}{a_2} $ erhalt man mit Hilfe des Satzes von Pythagoras. $|\vec{a}| = \sqrt{a_1^2+a_2^2}$

\hfill \break
Bei einem Vektor, der zwei Punkte verbindet, erhält man dadurch deren Entfernung (z.B.: 5 m oder 5 cm).
Bei Geschwindigkeiten oder Kräften erhält man ihren Betrag oder Wert (z.B.: 5 m/s oder 5 Newton).

\hfill \break
Die Länge eines Vektors kann man verändern, indem man ihn mit einem Skalar (= Zahl) multipliziert.
Man erhält dadurch parallele Vektoren, weil sie die gleiche Richtung haben.
$$k+\vec{a}=k*\binom{a_1}{a_2} = \binom{k*a_1}{k*a_2}$$

\hfill \break
Der Vektor $\vec{b}$ ist doppelt so lang wie der Vektor $\vec{a}$, hat aber die selbe Richtung und Orientierung.

\hfill \break
Soll die Orientierung des Vektors geändert werden, so multipliziert man ihn mit einer negativen Zahl.
Der Vektor $\vec{c}$ hat die selbe Richtung und Länge wie der Vektor $\vec{a}$, aber eine andere Orientierung.
Daher heißt er Gegenvektor des Vektors $\vec{a}$.

\hfill \break
Sucht man den Einheitsvektor, so sucht man einen Vektor mit der gleichen Richtung und Orientierung, aber mit der Länge 1.
Ihn erhält man, indem man den ursprünglichen Vektor durch seine Länge dividiert, oder mit dem Kehrwert seiner Länge multipliziert.
$$\vec{a_0}  = \frac{1}{|\vec{a}|}*\vec{a}$$

\hfill \break
Mit Einheitsvektoren kann man auch bestimmen, wie lange ein Vektor sein soll.
Hat er nämlich schon Länge 1, braucht er nur noch mit der gewünschten Länge multipliziert werden.