\subsection{Erwartungswert, Varianz und Standardabweichung einer 1200 cm Holzlatte }

Messserie mit 5 Daten 1201,1198,1202,1205,1194:\\
\begin{itemize}
    \item mittelwert: $m=\frac{1201+1198+1202+1205+1194}{5} = 1200$
    \item Streueung: $+1,-2,2,5-6$
    \item Quadrate der Abweichung: $(1,4,4,25,36)$
    \item Varianz:\begin{itemize}
        \item $s^2=\frac{70}{5} = 14$
        \item $s = \sqrt{14} = 3.74$
    \end{itemize}
    \item Streuungsbereich: $1200 \pm 3.7$
\end{itemize}

Die Streueung ist die Abweichung der einzenen Werte von zb: 1200 $(x-\mu)$.\\
Mittelwert der Abweichung in beiden Fällen O, daher Unbrauchbar-> Quadrate der Abweichungen $(x_1 -\mu)^2$.\\
Varianz Bezeichnent den Mittelwert der Quadrate der Abweichungen. 
