\subsection{Modus}

Der Modus einer Datenliste ist der am häufigsten vorkommende Weri der untersuchten Variablen.
Er wird vor allern bei qualitativen Variablen verwendet. Er ist nicht immer eindeutig bestimmt, da es mehrere häufigste Werte geben kann.

\hfill \break
Example:\\
Die Erhebung der Haarfarben einer Personengruppe liefert die Urliste: 
schwarz, blond, brünett, schwarz, blond, brünett, blond, blond, schwarz, blond, brünett, schwarz 
Der Modus ist hier blond. Würde man noch eine schwarzhaarige Person hinzunehmen, 
gäbe es zwei Modi, nämlich schwarz und blond. 

\hfill \break
Median (Zentralwert): Ordnet man eine liste $x_1,x_2 ... x_n$, von Zahlen der Größe nach, so heißt 
bei einer ungeraden Anzahl von Zahlen die in der Mitte stehende Zahl der Median der Liste, bei 
einer geraden Anzahl von Zahlen bezeichnet man das aritltrnetische Mittel der beiden in der 
Mitte stehenden Zahlen als den Median der Liste. In der Liste stehen vor und nach dem Median 
gleich viele Zahlen. 
