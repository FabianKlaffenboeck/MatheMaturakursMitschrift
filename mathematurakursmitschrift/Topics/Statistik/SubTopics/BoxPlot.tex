\newpage
\subsection{Box Plot}

In einem Wald wurden die Stammumfänge von 120 Bäumen vermessen (Umfang in cm). Die Daten sind hier in Form eines Diagramms dargestellt.

\hfill \break
\begin{center}
\resizebox{10cm}{5cm}{%
    \begin{tikzpicture}
        \begin{axis}
            [
                ytick={1},
                yticklabels={},
            ]
            \addplot+[
                boxplot prepared={
                        median=48,
                        upper quartile=60,
                        lower quartile=45,
                        upper whisker=67,
                        lower whisker=30
                    },
            ] coordinates {};
        \end{axis}
    \end{tikzpicture}
}

\hfill \break
\begin{itemize}
    \item ca. $75\%$ der Bäume einen Stammumfang kleiner als 60cm haben.
    \item alle Bäume einen Stammumfang von höchstens 67cm haben.
    \item von den 120 Bäumen ca. 3O einen Stammumfang von höchstens 45cm haben.
    \item ca.: 60 Bäume einen Stammumfang von 45 bis 60 cm haben
\end{itemize}
\end{center}