\newpage
\subsection{PQ und ABC Formel}


Um die abc-Formel anwenden zu können, müssen wir die quadratische Gleichung in die allgemeine Form überführen, das heißt dort muss $irgendwas = 0$ stehen. Liegt diese dann vor, können wir die abc-Formel direkt anwenden.
Sind 3 Terme vorhanden sprechen wir von der ABC Formel, wen nur 2 Terme vorhanden sind handelt es sich um die PQ Formal.
Diese kamm mit der ABC Formel leböst werden indem man die Fehlenden Terme mit 1 ersetzt.\\

\hfill \break
Example:\\
\fboxrule=0.8pt \fcolorbox{lightgray}{lightgray}{%
    \begin{tabular}{ |c|}
        \hline
        $3*x^2 + 3*x = 18$                                                                      \\
        $3*x^2 + 3*x = 18$  /-18                                                                \\
        $3*x^2 + 3*x - 18 = 0$                                                                  \\
        \\
        Nun wird die obige Formel herangezogen und eingesetzt. Es ist a = 3, b = 3 und c = -18. \\
        \\
        $x_{1,2} = \frac{-b\pm \sqrt{b^2-4*a*c}}{2*a}$  /a=3, b=3, c=-18                        \\
        \\
        $x_{1,2} = \frac{-3 \pm \sqrt{3^2 - 4*3*(-18)}}{2*3}$                                   \\
        \\
        $x_{1,2} = \frac{-3\pm\sqrt{9+216}}{6} $                                                \\
        \\
        $x_{1,2} = \frac{-3\pm\sqrt{225}}{6}$                                                   \\
        \\
        $x_{1,2} =\frac{-3\pm15}{6}$                                                            \\
        \\
        $x_1 = \frac{-3+15}{6} = \frac{12}{6} = 2$                                              \\
        \\
        $x_2 = \frac{-3-15}{6} = \frac{-18}{6} = -3$                                            \\
        \hline
    \end{tabular}}

\newpage
\hfill \break
Rechnungsweg im Taschenrechner $TI-82STATS$:\\
\begin{itemize}
    \item Wenn die ABC-Formel schon im Taschenrechner gespeichert ist:
    \item Taste $PRGM$
    \item ABC-Funktion auswählen
    \item $ENTER$ Taste
    \item Wenn nacheinander die Buchstaben A-B-C, erscheinen dann den entsprechenden Wert eintragen (Bei einer negativen Zahl das Vorzeichen verwenden und nicht das Rechenzeichen !!!)
    \item Danach erscheinen untereinander 2 Zahlen. Das sind die beiden Lösungen.
    \item Wenn die ABC-Formel noch nicht im Taschenrechner gespeichert ist, muss diese erst eingespeichert werden...
\end{itemize}
