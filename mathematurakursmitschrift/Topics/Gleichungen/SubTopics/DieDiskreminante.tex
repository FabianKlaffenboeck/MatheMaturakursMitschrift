\newpage
\subsection{Die Diskreminanten}

\hfill \break
Die Diskreminante gibt an wie viele Lösungen es in einer Quadratischen Gleichung gibt.
\hfill \break

Example:\\
\fboxrule=0.8pt \fcolorbox{lightgray}{lightgray}{%
    \begin{tabular}{c|c|c}
        $4x^2+x+10=0$                                & $4x^2-12x+9=0$                              & $2x^2+3x-20=0$                             \\
        \\
        $1X_2 = \frac{-1 \pm \sqrt{1-4*4*10}}{2*4} $ & $1X_2 = \frac{12 \pm \sqrt{144-4*4*9}}{8} $ & $1X_2 = \frac{-3 \pm \sqrt{9+4*2*20}}{4} $ \\
        \\
        $1X_2 = \frac{-1 \pm \sqrt{-159}}{2*4} $ & $1X_2 = \frac{12 \pm \sqrt{0}}{8} $         & $1X_2 = \frac{-3 \pm 13}{4} $              \\
        \\
        $ $                                          & $X_1= \frac{12}{8}= \frac{3}{2} $           & $X_1 =2.5  X_2 = -4 $                      \\
        $ D<0$                                       & $D=0 $                                      & $D>0 $                                     \\
        Keine Lösung                                 & Eine Lösung                                 & Zwei Lösungen                              \\
    \end{tabular}}