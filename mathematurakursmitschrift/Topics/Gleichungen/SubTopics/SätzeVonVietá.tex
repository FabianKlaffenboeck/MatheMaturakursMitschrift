\newpage
\subsection{Sätze von Vietá}

\hfill \break
Die Sätze von Vietá werden verwendet um $p$ und $q$ zu errechnen.

\hfill \break
\begin{itemize}
    \item (1.): $(x-x_1)*(x-x_2) = 0$
    \item (2.): $x_1*x_2 = q$
    \item (3.): $x_1+x_2 = -p$ oder $-(x_1+x_2) = p$
\end{itemize}

\hfill \break
Example:\\
\fboxrule=0.8pt \fcolorbox{lightgray}{lightgray}{%
    \begin{tabular}{ |c|}
        \hline
        $(x-3)*(x+7)=0$ \\
        $\Downarrow \Downarrow $ \\
        $x_1 = 3$    $x_2 = -7$ \\
        $x^2-3x+7x-21=0$\\
        $x^2+4x-21=0$\\
        $\Downarrow$\\
        $x^2+4x-21=0$\\
        $x^2+px+q=0$\\
        \hline
    \end{tabular}}