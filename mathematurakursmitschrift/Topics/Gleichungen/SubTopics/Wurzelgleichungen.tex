\newpage
\subsection{Wurzelgleichungen}

\hfill \break
Von einer Wurzelgleichung spricht man wen Teile der Gleichung unter einer Wurzel stehen.
Die Probe ist Umungänglich da durch das Quadrieren (da das keine Eqivalenzumformung ist) eine "falsche Lösung" entstehen kann.

\hfill \break
Example einfache Wurzelgleichungen:\\
\fboxrule=0.8pt \fcolorbox{black}{lightgray}{%
    \begin{tabular}[t]{@{}l@{}}
        Rechnung:                 \\
        $\sqrt{2x+3} = 5$ // $^2$ \\
        $2x+3 = 25$ // $-3$       \\
        $2x = 22$ // $/2$         \\
        $x = 11$                  \\
        \\
        Probe:                    \\
        $\sqrt{2*11+3} = 5$       \\
        $\sqrt{25} = 5$           \\
        $5 = 5 = w.A.$            \\
    \end{tabular}}

\hfill \break
Example Wurzelgleichung mit Binomischer Formel:\\
\fboxrule=0.8pt \fcolorbox{black}{lightgray}{%
    \begin{tabular}[t]{@{}l@{}}
        Rechnung:                            \\
        $\sqrt{2x+7} = \sqrt{x+3}+1$ // $^2$ \\
        $2x+7 = (x+3)+2*1*\sqrt{x+3}+1$      \\
        $2x+7 = x+4+2*\sqrt{x+3}$ // $-x-4$  \\
        $x+3 = 2\sqrt{x+3}$ // $^2$          \\
        $x^2+6x+9 = 4x+12$ //$-4x-12$        \\
        $x^2+2x-3 = 0$                       \\
        $x_1 = 1$                            \\
        $x_2 = -3$                           \\
        \\
        Probe  1:                            \\
        $\sqrt{2-7} = \sqrt{1-3}+1$          \\
        $3 = 3 = w.A.$                       \\
        \\
        Probe  2:                            \\
        $\sqrt{-6+7} = \sqrt{-3+3}+1$        \\
        $1 = 1 = w.A.$                       \\
    \end{tabular}}