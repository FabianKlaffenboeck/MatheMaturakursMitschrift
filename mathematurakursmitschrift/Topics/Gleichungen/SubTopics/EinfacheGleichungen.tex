\subsection{Einfache Gleichungen}

Ich denke mir eine Zahl, multipliziere sie mit 2 und subtrahiere anschliesend 1. Ich erhalte 15.

\hfill \break
Example:\\
\fboxrule=0.8pt \fcolorbox{lightgray}{lightgray}{%
    \begin{tabular}{ |c|}
        \hline
        $2x-1=15$ / +1 \\
        $2x=16$ / /2   \\
        $x=8$          \\
        \hline
    \end{tabular}}\\

\hfill \break
Example:\\
\fboxrule=0.8pt \fcolorbox{lightgray}{lightgray}{%
    \begin{tabular}{ |c|}
        \hline
        $(x-9)*4=100$ / Ausmultiplizieren \\
        $4x-36=100$ / +36                 \\
        $4x=136$ / /4                     \\
        $x=34$                            \\
        \hline
    \end{tabular}}\\

\hfill \break
Bei einem Minus vor der Klammer werden alle Vorzeichen in der Klammer vertauscht.\\
Example:\\
\fboxrule=0.8pt \fcolorbox{lightgray}{lightgray}{%
    \begin{tabular}{ |c|}
        \hline
        $3(x+4)-(x-8) = 6x-2(3+x)$ / Klammern auflösen \\
        $3x+12-x+8=6x-6-2x$ / Zusammenrechnen          \\
        $2x+20=4x-6$ / -2x                             \\
        $20=2x-6$ / +6                                 \\
        $26=2x$ / /2                                   \\
        $13=x$                                         \\
        $x=13$                                         \\
        \hline
    \end{tabular}}
