\hfill \break
\subsection{Gleichungen mit Brüchen}

\hfill \break
Example:\\
\fboxrule=0.8pt \fcolorbox{black}{lightgray}{%
    \begin{tabular}[t]{@{}l@{}}
        $\frac{x}{2} +\frac{x}{3} = 5$ /gemeinsamer Nenner ist 6 \\
        $\frac{3x+2x}{6} = 5$                                    \\
        $\frac{5x}{6} = 5$ / *6                                  \\
        $5x = 30$ / /5                                           \\
        $x = 6$                                                  \\
    \end{tabular}}\\

\hfill \break
Example:\\
\fboxrule=0.8pt \fcolorbox{black}{lightgray}{%
    \begin{tabular}[t]{@{}l@{}}
        $\frac{2x}{3} -\frac{x}{4} = \frac{5}{6}$ /gemeinsamer Nenner ist 12 \\
        \hfill                                                               \\
        $\frac{8x-3x}{12} = \frac{10}{12}$                                   \\
        \hfill                                                               \\
        $\frac{5x}{12} = \frac{10}{12}$ / *12                                \\
        $5x = 10$ / /5                                                       \\
        $x = 2$                                                              \\
    \end{tabular}}