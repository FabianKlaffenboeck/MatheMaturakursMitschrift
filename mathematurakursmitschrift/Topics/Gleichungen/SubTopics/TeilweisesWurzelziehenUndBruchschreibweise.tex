\newpage
\subsection{Teilweises Wurzelziehen und Bruchschreibweise}


\hfill \break
Beim teilweisen Wurzelziehen wird nur von einem Faktor die Wurzel gezogen. Der andere Faktor bleib unter der Wurzel stehen.\\

\hfill \break
Regeln:\\
\begin{itemize}
    \item $\sqrt{a}* \sqrt{b} = \sqrt{ab}$
    \item $\sqrt[n]{a}* \sqrt[n]{b} = \sqrt[n]{ab}$
    \item $\sqrt{9}+ \sqrt{16} \neq  \sqrt{9+16}$
    \item $3+5 \neq 5$
    \item $\frac{\sqrt[n]{a}}{\sqrt[n]{b}} = \sqrt[n]{\frac{a}{b}}$
    \item $a^\frac{r}{s} \Leftrightarrow  \sqrt[s]{a^r}$
    \item $a^{-\frac{r}{s}} \Leftrightarrow  \frac{1}{\sqrt[s]{a^r}}$
\end{itemize}

\hfill \break
Example:\\
\fboxrule=0.8pt \fcolorbox{black}{lightgray}{%
    \begin{tabular}[t]{@{}l@{}}
        $\sqrt{12} = \sqrt{4} * \sqrt{3} = 2* \sqrt{3}$      \\
        $\sqrt{50} = \sqrt{25} * \sqrt{2} = 5* \sqrt{2}$     \\
        $\sqrt{18} = \sqrt{9} * \sqrt{2} =3* \sqrt{2}$       \\
        \\
        $\sqrt[2]{a^2} = a$                                  \\
        $\sqrt[2]{a^3} = a*\sqrt{2}$                         \\
        $\sqrt[2]{a^5} = a^2*\sqrt{2}$                       \\
        $\sqrt{50*a^4*b^5*c^7} = 5*a^2*b^2*c^3*\sqrt{2*b*c}$ \\
        \\
        $a^\frac{2}{5} = \sqrt[5]{a^2}$                      \\
        $a^{-\frac{3}{7}} = \frac{1}{\sqrt[7]{a^3}}$         \\
    \end{tabular}}