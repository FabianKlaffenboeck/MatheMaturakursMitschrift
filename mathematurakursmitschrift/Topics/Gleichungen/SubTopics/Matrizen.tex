\newpage
\subsection{Matrizen rechnen}

\hfill \break
Eine Matrix (Gleichungssysteme mit 3 Unbekannten), die aus $m$ Zeilen und $n$ Spalten besteht, hat die Dimension $m\times$.

\hfill \break
\[
    A_{m\times n} =
    \left[ {\begin{array}{cccc}
                    a_{11} & a_{12} & \cdots & a_{1n} \\
                    a_{21} & a_{22} & \cdots & a_{2n} \\
                    \vdots & \vdots & \ddots & \vdots \\
                    a_{m1} & a_{m2} & \cdots & a_{mn} \\
                \end{array} } \right]
\]

\hfill \break
Example:\\
\fboxrule=0.8pt \fcolorbox{lightgray}{lightgray}{%
    \begin{tabular}{ |c|}
        \hline
        (1)$x+2y+3z=10$ \\
        (2)$2x+3y+z=13$ \\
        (3)$3x+y+2z=13$ \\
        \hline
    \end{tabular}}\\

\hfill \break
Rechenmatrix:
$\left(\begin{array}{cccc}
            1 & 2 & 3 \\
            2 & 3 & 1 \\
            3 & 1 & 2 \\
        \end{array}\right)$\\

\hfill \break
Lösung:
$\left(\begin{array}{ccccc}
            10 \\
            13 \\
            13 \\
        \end{array}\right)$\\

\break
Berechnen einer Matrix mit dem Taschenrechner $TI-82STATS$:\\
\begin{itemize}
    \item Taste $MATRX$
    \item Taste 2*$\rightarrow$ Nach rechts auf Edit
    \item Wenn schon die gewünschte Matrix vorhanden ist diese auswählen ansonsten Taste $Enter$
    \item Wenn neue Matrix, dann größe auswählen z.b. 3X4
    \item oben links in der Ecke anfangen die Zahlen einzugeben. Wenn in der maxtrix keine zahl steht dann$1$ eigeben.
    \item Wenn alle Zahlen eingegeben sind mit $^2nd$ $Quit$ aus dem Menü herrausgehen.
    \item Taste  $MATRX$
    \item Taste $\rightarrow$ Nach rechts auf Math
    \item Taste 5*$\uparrow$ auf "rref("
    \item Taste $ENTER$
    \item Taste $MATRX$
    \item gewünschte Matrix auswählen
    \item Taste $ENTER$
    \item das ergebnis steht in der letzten Spalte ganz rechts
\end{itemize}