\newpage
\subsection{Exponentialgleichungen}

\hfill \break
Beispiele:
\begin{itemize}
    \item $2^x = 4$ $\rightarrow$ $x = 2$
    \item $2^x = 8$ $\rightarrow$ $x = 3$
    \item $2^x = 6$ $\rightarrow$ $x = ?$
\end{itemize}

\hfill \break
Die eindeutige Lösung von $a^x = b$ nennt man den Logarithmus von b zur Basis a.\\
$a^x = b$ $\Leftrightarrow$ $Log_a(b) = x$


\hfill \break
Example:\\
\fboxrule=0.8pt \fcolorbox{black}{lightgray}{%
    \begin{tabular}[t]{@{}l@{}}
        $2^x = 6$ // Log()                \\
        \\
        $Log(2^x)$ = $Log(6)$             \\
        \\
        $x * Log(2)$ = $Log(6)$ // Log(2) \\
        \\
        $x = \frac{Log(6)}{Log(2)}$       \\
        \\
        $x = 2.584962501$
    \end{tabular}}