\newpage
\subsection{Formelumformen}

\hfill \break
Example auflösen nach $k$:\\
\fboxrule=0.8pt \fcolorbox{black}{lightgray}{%
    \begin{tabular}[t]{@{}l@{}}
        $-8=4k$  / /4 \\
        $-2=k$        \\
    \end{tabular}}

\hfill \break
Example auflösen nach $d$:\\
\fboxrule=0.8pt \fcolorbox{black}{lightgray}{%
    \begin{tabular}[t]{@{}l@{}}
        $6=(-1)*(-2)+d$ \\
        $6=2+d$ /-2     \\
        $d=4$           \\
    \end{tabular}}

\hfill \break
Example auflösen nach $k$:\\
\fboxrule=0.8pt \fcolorbox{black}{lightgray}{%
    \begin{tabular}[t]{@{}l@{}}
        $\frac{1}{e}=\frac{1}{g}+\frac{1}{k}$ / auf gemeisamen Nenner bringen (egk) \\
        $\frac{gk}{egk}=\frac{ek+eg}{egk}$ /*egk                                    \\
        $gk=ek+eg$ /-ek                                                             \\
        $gk-ek=eg$                                                                  \\
        $k*(g-e)=eg$ //(e-g)                                                        \\
        $k=\frac{eg}{g-e}$                                                          \\
    \end{tabular}}

\hfill \break
Example auflösen nach $m$:\\
\fboxrule=0.8pt \fcolorbox{black}{lightgray}{%
    \begin{tabular}[t]{@{}l@{}}
        $m=\frac{1-k}{b+k}$ /*(b+k) \\
        $m*(b+k)=1-k$               \\
        $mb+mk=1-k$ /-1\ /-mk       \\
        $mb-1=k-mk$                 \\
        $mb-1=k*(-1-mk)$            \\
        $k=\frac{mb-1}{1-(1+m)}$    \\
        $k=\frac{1-md}{m+1}$        \\
    \end{tabular}}\\