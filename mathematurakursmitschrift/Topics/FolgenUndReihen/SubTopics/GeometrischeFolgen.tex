\newpage
\subsection{Geometrische Folgen}

Eine Folge heißt geometrisch, wenn der Quotient $q$ zweier aufeinander folgender Glieder stets konstant ist.
Für eine jede geometrische Folge gilt also:

$$\frac{a_{\mathrm{n + 1}}}{ a_{\mathrm{n}} } = q$$

Als Bildungsgesetz gilt:

$$a_{\mathrm{n}} =  a_1 \cdot q ^{n-1}$$

Ist $q > 1$, so ist die Folge (streng) monoton zunehmend, bei $0 < q< 1$ ist die Folge (streng) monoton abnehmend und konvergiert gegen Null.
Gilt $q=0$, so ist die Folge konstant, im Fall - $\infty < q < 0$ ist die Folge alternierend, die Werte der Folgenglieder sind also abwechselnd positiv und negativ.

Da die einzelnen Folgenglieder immer um den gleichen Faktor zu- beziehungsweise abnehmen, ist das mittlere dreier Folgenglieder stets gleich dem geometrischen Mittel der beiden benachbarten Folgenglieder. Es gilt also:[2]

$$| a_{\mathrm{n}} | = \sqrt{a_{\mathrm{n+1}} \cdot a_{\mathrm{n-1}}}$$

Will man zwischen zwei Werten $a_1$ und $a_2$ insgesamt n weitere Zahlen als eine geometrische Folge einfügen, so gilt dabei für alle Quotienten der einzelnen Folgenglieder:

$$q_{\mathrm{i}} = \sqrt[n+1]{\frac{ a_2}{ a_1}}$$