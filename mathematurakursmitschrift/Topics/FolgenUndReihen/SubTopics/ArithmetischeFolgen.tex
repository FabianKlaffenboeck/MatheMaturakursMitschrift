\newpage
\subsection{Arithmetische Folgen}

Eine Folge heißt arithmetisch, wenn die Differenz d zweier aufeinander folgender Glieder stets konstant ist.
Für eine arithmetische Folge gilt also:

$$a_{\mathrm{n + 1}} - a_{\mathrm{n}} = d$$

Als Bildungsgesetz gilt:

$$a_{\mathrm{n}} =  a_1 + (n - 1) \cdot d$$

Ist $d > 0$, so ist die Folge (streng) monoton steigend, bei $d < 0$ ist die Folge (streng) monoton fallend. Gilt $d=0$, so ist die Folge konstant.

Da die einzelnen Folgenglieder immer um den gleichen Betrag zu- beziehungsweise abnehmen, ist das mittlere dreier Folgenglieder stets gleich dem arithmetischen Mittel der beiden benachbarten Folgenglieder.
Es gilt also:

$$a_{\mathrm{n}} = \frac{a_{\mathrm{n + 1}} + a_{\mathrm{n-1}}}{2}$$

Wichtige arithmetische Folgen sind beispielsweise die natürlichen Zahlen $1 ,\, 2 ,\, 3 ,\, 4 ,\, \ldots$,
die geraden Zahlen $2 ,\, 4 ,\, 6 ,\, 8 ,\, \ldots$, die ungeraden Zahlen $1 ,\, 3 ,\, 5 ,\, 7 ,\,\ldots$, usw.

Will man zwischen zwei Werten $a_1$ und $a_2$ insgesamt $n$ weitere Zahlen als eine arithmetische Folge einfügen, so gilt dabei für alle Differenzen der einzelnen Folgenglieder:

$$d_{\mathrm{i}} = \frac{a_2 - a_1}{n + 1}$$

Diese Formel kann beispielsweise hilfreich sein, um fehlende Werte in Wertetabellen (näherungsweise) zu ergänzen.
Eine ähnliche Anwendung kann darin bestehen, $n$ Objekte (beispielsweise Holzbalken) in jeweils gleichem Abstand voneinander zwischen zwei festen Grenzen $a_1$ und $a_2$ einzufügen; dabei gibt $d_{\mathrm{i}}$ an, in welchem Abstand die Mittelpunkte der Objekte jeweils eingefügt werden müssen.