\newpage
\section{Folgen und Reihen}

Ordnet man jeder natürlichen Zahl $n \in \mathbb{N}$ eine reelle Zahl $a_{\mathrm{n}}$ eindeutig zu, so entsteht eine unendliche (reelle) Folge $(a_{\mathrm{n}})$.
Die einzelnen Werte der Folge heißen Folgenglieder und werden mit Indizes durchnummeriert:

$$( a_{\mathrm{n}} ) = a_1 ,\,  a_2 ,\, a_3 ,\, \ldots,\, a_{\mathrm{n}} ,\, \ldots$$

Im Unterschied zu einer Menge kann bei einer Folge ein und das selbe Glied mehrere Male auftreten.
Die Definition einer Folge kann auf zweierlei Arten erfolgen:

Viele Folgen lassen sich nach einem Bildungsgesetz mittels eines Terms aufstellen. Das Bildungsgesetz wird hierzu in runde Klammern geschrieben.
Beispiel:

$$(a_{\mathrm{n}}) = (2 \cdot n^2) = 2 ,\,  8 ,\,  18 ,\, 32 ,\, \ldots$$

Ist (mindestens) das erste Folgenglied bekannt und besteht eine Rechenvorschrift, wie sich ein Folgenglied aus einem vorhergehenden berechnen lässt, so sind alle Glieder einer Folge ebenfalls eindeutig festgelegt.
Dieses Vorgehen wird als Rekursion bezeichnet. Beispiel:

$$a_{\mathrm{n}} = 0 ,\, 1 ,\, 2 ,\, 3 ,\, 5 ,\, 8 ,\, 13 ,\, 21 ,\, \ldots$$

Die obige Zahlenfolge wird auch zu Ehren von Leonardo Fibonacci als Fibonacci-Folge bezeichnet.
Die Folgenglieder lassen sich dadurch berechnen, indem jeweils die Summe der beiden vorangehenden Folgenglieder gebildet wird.
Das Bildungsgesetz der Folge lautet somit für $n \ge 2$:

$$a_{\mathrm{n}} = a_{\mathrm{n-2}} + a_{\mathrm{n-1}}$$

Beschränkt man die Definitionsmenge auf die ersten n natürlichen Zahlen $(n \ne 0)$, so erhält man eine endliche Folge mit dem Anfangsglied $a_1$ und dem Endglied $a_{\mathrm{n}}$.


\hfill \break
\newpage
\subsection{Arithmetische Folgen}

Eine Folge heißt arithmetisch, wenn die Differenz d zweier aufeinander folgender Glieder stets konstant ist.
Für eine arithmetische Folge gilt also:

$$a_{\mathrm{n + 1}} - a_{\mathrm{n}} = d$$

Als Bildungsgesetz gilt:

$$a_{\mathrm{n}} =  a_1 + (n - 1) \cdot d$$

Ist $d > 0$, so ist die Folge (streng) monoton steigend, bei $d < 0$ ist die Folge (streng) monoton fallend. Gilt $d=0$, so ist die Folge konstant.

Da die einzelnen Folgenglieder immer um den gleichen Betrag zu- beziehungsweise abnehmen, ist das mittlere dreier Folgenglieder stets gleich dem arithmetischen Mittel der beiden benachbarten Folgenglieder.
Es gilt also:

$$a_{\mathrm{n}} = \frac{a_{\mathrm{n + 1}} + a_{\mathrm{n-1}}}{2}$$

Wichtige arithmetische Folgen sind beispielsweise die natürlichen Zahlen $1 ,\, 2 ,\, 3 ,\, 4 ,\, \ldots$,
die geraden Zahlen $2 ,\, 4 ,\, 6 ,\, 8 ,\, \ldots$, die ungeraden Zahlen $1 ,\, 3 ,\, 5 ,\, 7 ,\,\ldots$, usw.

Will man zwischen zwei Werten $a_1$ und $a_2$ insgesamt $n$ weitere Zahlen als eine arithmetische Folge einfügen, so gilt dabei für alle Differenzen der einzelnen Folgenglieder:

$$d_{\mathrm{i}} = \frac{a_2 - a_1}{n + 1}$$

Diese Formel kann beispielsweise hilfreich sein, um fehlende Werte in Wertetabellen (näherungsweise) zu ergänzen.
Eine ähnliche Anwendung kann darin bestehen, $n$ Objekte (beispielsweise Holzbalken) in jeweils gleichem Abstand voneinander zwischen zwei festen Grenzen $a_1$ und $a_2$ einzufügen; dabei gibt $d_{\mathrm{i}}$ an, in welchem Abstand die Mittelpunkte der Objekte jeweils eingefügt werden müssen.
\hfill \break
\newpage
\subsection{Geometrische Folgen}

Eine Folge heißt geometrisch, wenn der Quotient $q$ zweier aufeinander folgender Glieder stets konstant ist.
Für eine jede geometrische Folge gilt also:

$$\frac{a_{\mathrm{n + 1}}}{ a_{\mathrm{n}} } = q$$

Als Bildungsgesetz gilt:

$$a_{\mathrm{n}} =  a_1 \cdot q ^{n-1}$$

Ist $q > 1$, so ist die Folge (streng) monoton zunehmend, bei $0 < q< 1$ ist die Folge (streng) monoton abnehmend und konvergiert gegen Null.
Gilt $q=0$, so ist die Folge konstant, im Fall - $\infty < q < 0$ ist die Folge alternierend, die Werte der Folgenglieder sind also abwechselnd positiv und negativ.

Da die einzelnen Folgenglieder immer um den gleichen Faktor zu- beziehungsweise abnehmen, ist das mittlere dreier Folgenglieder stets gleich dem geometrischen Mittel der beiden benachbarten Folgenglieder. Es gilt also:[2]

$$| a_{\mathrm{n}} | = \sqrt{a_{\mathrm{n+1}} \cdot a_{\mathrm{n-1}}}$$

Will man zwischen zwei Werten $a_1$ und $a_2$ insgesamt n weitere Zahlen als eine geometrische Folge einfügen, so gilt dabei für alle Quotienten der einzelnen Folgenglieder:

$$q_{\mathrm{i}} = \sqrt[n+1]{\frac{ a_2}{ a_1}}$$
