\newpage
\section{Kostenrechnung}

\hfill \break
Kostenfuntion:
\begin{itemize}
    \item Lineare Kostenfuntion: $K(x)=30x+\textcolor{green}{2500}$
    \item Quadratische Kostenfuntion: $K(x)=0.0.x^2+120x+\textcolor{green}{12800}$
    \item Kostenfuntion 3.Grades: $K(x)=-0.05x^3+7.775x^2+-10.31x+\textcolor{green}{5000}$
\end{itemize}

\begin{itemize}
    \item $\textcolor{green}{Zusatzkosten}$
\end{itemize}


\hfill \break
Der Wendepunkt in einer Kostenrechnung wird Kostenkeere gennnat.
\begin{itemize}
    \item bis zu diesem Punkt: Degresive Kosten
    \item ab dem Punkt: Progresive Kosten
\end{itemize}

\hfill \break
Duchschnittliche Kosten pro Stück: $\frac{K(x)}{x}$\\

\hfill \break
Erlösfunktion (Umsatz): $E(x) = \textcolor{red}{p}*\textcolor{blue}{t}$\\
\begin{itemize}
    \item $\textcolor{red}{p}$ = Verkaufspreis
    \item $\textcolor{blue}{t}$ = Stück
\end{itemize}

\hfill \break
Gewinnfuntion: $G(x) = E(x)-K(x)$\\
$G(x)=0$ = Break even point / Gewinnschwelle

\hfill \break
Nachfragefuntion: $n(x) = -1.5x+360$\\
Preisefuntion: $p(x) = \frac{E(x)}{x}$

\hfill \break
Sättigungsmenge: Jene Stückzahl wo der Preisen gleich 0 ist. (Markt ist Gesättigt)\\
Hüchstpreich: Jener Preis bei dem kein eiziges Stück mehr verkauft wird.