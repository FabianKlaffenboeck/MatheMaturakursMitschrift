\subsubsection{Beispiel 2: Der Antennemast}

\hfill \break
Der Antennenmast eines Fernsehturms hat die Höhe h = 75m. Von einem Geländepunkt P werden Spitze und Fußpunkt
des Antennenmast unter den Höhenwinkeln a = 24°12' und ß = 17°42' gesehen. Ermittle die Höhe des Fernsehturms
mit Sendemast.
(Lösung: 258,73m)

\hfill \break
\begin{itemize}
    \item Winkel $\alpha = 24.2$°
    \item Winkel $\beta = 17+\frac{42}{60}$°
    \item Winkel $\gamma = 24.2-1.7=6.5$°
    \item Winkel $\delta = 180 - 90 - 17.7 = 72.3$°
    \item Winkel $\epsilon = 107.7$°
    \item Winkel $\zeta = 65.8$°
\end{itemize}

\hfill \break
Rechenweg:\\
\fboxrule=0.8pt \fcolorbox{black}{lightgray}{%
    \begin{tabular}[t]{@{}l@{}}
        $\frac{75}{Sin(\alpha)} = \frac{\gamma}{Sin(\zeta)}$ \\
        $\frac{75}{Sin(\alpha)} = \gamma$                    \\
        $y = 604.3$                                          \\
        \\
        $Sin(\beta) = \frac{x}{y}$                           \\
        $Sin(\beta)*\gamma = x$                              \\
        $x = 183.73$                                         \\
        \\
        $H = 183.73 + 75 = 258.73$                           \\
    \end{tabular}}

%//TODO add drawing