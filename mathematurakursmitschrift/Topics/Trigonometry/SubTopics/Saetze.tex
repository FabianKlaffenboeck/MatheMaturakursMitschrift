\newpage
\subsection{Sinus und Kosinussatz und Fläche}

\hfill \break
In jedem Dreieck verhalten sich die Längen zweier Seiten wie die Sienuswerte der gegenüberliegenden Winkel.

\hfill \break
\begin{itemize}
    \item 1.Sinussatz $\rightarrow \frac{sin(\alpha)}{a}$
    \item 2.Sinussatz $\rightarrow \frac{sin(\beta)}{b}$
    \item 3.Sinussatz $\rightarrow \frac{sin(\gamma)}{c}$
    \item 1.Kosinussatz $\rightarrow c^2 = a^2 + b^2 +2*a*b*cos(\gamma)$
    \item 2.Kosinussatz $\rightarrow b^2 = a^2 + c^2 +2*a*c*cos(\beta)$
    \item 3.Kosinussatz $\rightarrow a^2 = b^2 + c^2 +2*b*c*cos(\alpha)$
    \item Fläche im algemeinen Dreieck $\rightarrow A = \frac{a*ha}{2}$
    \item Fläche im algemeinen Dreieck $\rightarrow A = \frac{c*b*sin(\alpha)}{2}$
\end{itemize}
