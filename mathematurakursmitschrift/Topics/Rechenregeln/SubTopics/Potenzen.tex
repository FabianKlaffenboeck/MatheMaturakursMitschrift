\subsection{Potenzen}

\begin{huge}
    $a^n$
\end{huge}

\begin{itemize}
    \item Alles zusammen wird als Potenz bezeichnet
    \item a wird als Basis bezeichnet
    \item n wird als Exponent bezeichnet
\end{itemize}

\hfill \break
Example:\\
\fboxrule=0.8pt \fcolorbox{black}{lightgray}{%
    \begin{tabular}[t]{@{}l@{}}
        $10^2 = $ = 10*10 = 100                \\
        $10^3 = $ = 10*10*10 = 1000            \\
        $(-5)^3$ = $(-5)$*$(-5)$*$(-5)$ = -125 \\
        $(-2)^4$ = $(-2)$*$(-2)$*$(-2)$*$(-2)$ = 16
    \end{tabular}}\\


\hfill \break
Ist die Basis eine negative Zahl und der Exponent,...\\
\begin{tabular}[t]{@{}l@{}}
    ...Gerade so ist das Ergebnis positiv. \\
    ...ungerade so ist das Ergebnis negativ.
\end{tabular}